\section{Methodology}
\subsection{Atmospheric neutrinos}
Atmosperic neutrinos are produced by the interaction of cosmic rays with the upper atmosphere. In the subsequent calculations we used the 3D Monte Carlo calculations by Honda \textit{et al.}~\cite{Honda:2011nf}. This model is based on the nuclear interaction model JAM used in Particle and Heavy-Ion Transport code System and modified DPMJET-III package. For neutrino fluxes with energy above 10\,GeV we use spectra by Sinegovskaya \textit{et al.}~\cite{Sinegovskaya:2014pia} which is computed by using Hillas-Gaisser cosmic ray spectra~\cite{Gaisser:2013ira} and Kimel-Mokhov hadronic model~\cite{Kalinovsky:1989kk} in semianalytical approach. Figures represent energy and zenith-angle distributions of atmospheric neutrino fluxes used in this paper.

\subsection{Neutrino oscillations in medium}
Atmosperic neutrino fluxes are modified by neutrino oscillation phenomenon. For computing the flavor transition and survival probabilities we applied the result of the global neutrino oscillation data analysis (values of the neutrino mixing angles, CP violating phase and squared mass splittings) by Tortola \textit{et al.}~\cite{Tortola:2012te}.

Before getting the detector, neutrinos pass through the body of the Earth. Therefore, we had to take into account the neutrino-matter coherent scattering (MSW effect). To solve the Wolfenstein equations describing the evolution of the neutrino system in a medium we decided to adopt the method described in work~\cite{Naumov:2001ci}.

The density profile in the Earth in the presented calculations is described according to the Preliminary Reference Earth Model (PREM)~\cite{Dziewonski:1981xy} which contains 10 layers of density. According to the method, each of the layers with varying density must be split into a number of sublayers of conditionally constant density. Hereby the process of neutrino propagating, from the moment of its birth in the sourse to the moment of entering the detector, may be considered as a process of consecutive propagations through a number of segments of relatively invariable density. Thus if ${S_{i}}$ is the evolution operator and the solution of MSW equation for the system of three mixed Dirac neutrinos with the fixed flavours ($\nu_{e}$,$\nu_{\mu}$,$\nu_{\tau}$) passing through the $i$-th segment of constant density then $S=\prod{S_{i}}$ is the solution for the whole path from the source to the detector.

The convinience of such method follows from the fact that independently of density distribution in a medium we have to find an exact solution of the equation for constant density only. Such solution has been obtained in work~\cite{Naumov:1991ju,Naumov:1991rh}. It is also worth noticing here that the number of sublayers in each layer is, by virtue of different thickness of such layers, individual and regulated by a convergence criterion for the method used. The criterion itself states that there're numbers ${N_{i}}$ of segments that each $i$-th layer is split into such that further increasing of these numbers doesn't change the values of probabilities within the desired accuracy. In our case these numbers can be established numerically only.

\subsection{Effective axial mass approach}
The problem with the nucleon axial mass value is in a huge disagreement between the high-energy and free-nucleon best-fit value of $M_A$ and the $M_A$ extracted in K2K and (especially) MiniBooNE and T2K experiments.

Currently there is no generally accepted model for calculating neutrino-nucleus interaction cross sections, applicable in a wide energy range from threshold to ultrahigh values. The importance of the choice of nuclear physics model for neutrino oscillations is considered in the work by Meloni and Martini~\cite{Meloni:2012fq}, where the results of the experiment T2K~\cite{Abe:2011sj,Abe:2012gx} retrieves the value of the neutrino mixing parameters with using conventional Fermi gas model, as well as MECM model~\cite{Martini:2009uj}. The most striking difference is observed for the angle $\theta_{13}$: $\sin^{2}2\theta_{13}^{\textrm{RFG}}=0.138^{+0.031}_{-0.041}$ is approximately 2 times bigger than $\sin^{2}2\theta_{13}^{\textrm{MECM}}=0.092^{+0.030}_{-0.052}$.







































