For the accurate interpretation of neutrino experiment results, in particular, for the neutrino mixing parameter extraction, it is necessary to study theoretical uncertainty in predicted count rates in detectors.

Quasielastic neutrino-nucleus interaction makes the main contribution to the count rate of so-called muon-like and electron-like events identified as single-ring events fully contained in water-Cherenckov detectors like Super-Kamiokande. Also quasielastic scattering events can be identified separately in tracking detectors, e.g. NO$\nu$A.

An essential source of calculation uncertainty belongs to cross sections of neutrino interactions with nucleons and nuclei. There is currently no generally accepted single nuclear model, applicable in a wide energy range, from threshold to extremely high values. Also there is a problem with nucleon structure functions, especially, with so-called nucleon axial mass, an important phenomenological parameter characterizing the axial-vector and pseudoscalar form factors. The range of quasielastic axial mass values used in the current Monte Carlo neutrino generators for a number of experiments (from 0.99 GeV in T2K to 1.2 GeV in Super-Kamiokande) seems to be unreasonably wide and distorting the interpretation of the experimental results.

Phenomenological solution for both these problems is to apply the effective axial mass, which means the use of the relativistic Fermi gas model of the nucleus at the same time with the introduction of the axial form factor parameterization. From the statistical analysis of available accelerator data on quasielastic neutrino scattering we extract the shape of the parameterization and optimal values of the parameters, which can be recommended for use in the neutrino generators.

\textbf{Brief alternative version}

In this study, we estimate the error in quasielastic event rate predictions for atmospheric and accelerator neutrino experiments caused by the uncertainty in the nucleon axial mass value. It follows from our analysis that the impact of the axial mass uncertainty is comparable in magnitude with the expected neutrino oscillation effect itself. We propose a simple phenomenological method which allows to describe experimental quasielastic neutrino interaction data consistently and decrease uncertainty in the extracted neutrino mixing parameters.
