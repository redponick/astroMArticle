\section{Conclusions}
In this paper we demonstrated that the choice of the quasielastic nucleon axial mass value essentially affects the predicted count rates in neutrino detectors. $M_{A}^{\mathrm{eff}}$ emploiment, which described neutrino cross section data in various ranges of energy, predicts distinction which cannot be eliminated by any renormalizing factor.

The related uncertainty is comparable with that in the predicted atmospheric neutrino fluxes, as well as with the scale of the oscillation effect itself. Hence, the effect should be taken into account for the correct determination of the neutrino mixing parameters. 

We believe that using the effective axial mass instead of the value $1.2$\,GeV currently used by the SK Collaboration~\cite{Wendell:2010md} or the conventional constant value $M_{A}\approx$\,GeV should substantially decrease the uncertainty of the extracted oscillation parameters.

\section*{Acknowledgements}
We thank A.\/S.~Sheshukov for the help in the work.
