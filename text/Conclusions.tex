\section{Conclusions}
In this paper we demonstrated that the choice of the nucleon axial mass value essentially affects the predicted count rates in the Super-Kamiokande detector particularly within the region, which the QES interactions are dominant in. The related uncertainty is comparable with that in the predicted atmospheric neutrino fluxes, as well as with the scale of the oscillation effect itself. Hence, the effect should be taken into account for the correct determination of the neutrino mixing parameters. 

Let us emphasize the fact that the estimations presented herein are preliminary. However, we believe that using the effective axial mass instead of the value 1.2\,GeV currently used by the SK Collaboration~\cite{Wendell:2010md} or the value $0.99$\,GeV used by the T2K Collaboration~\cite{Abe:2011ks} should substantially improve the validity of the extracted neutrino mixing parameter values.

Future progress in the nuclear modelling and new experiments (e.g., MINER$\nu$A) will hopefully allow us to improve the accuracy of the current $M_A$ extraction and $M_{A}^{\mathrm{eff}}$ determination.
