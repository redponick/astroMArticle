\section{Methodology}
\subsection{Atmospheric neutrinos}
Atmosperic neutrinos are produced by the interaction of cosmic rays with the upper atmosphere. In the subsequent calculations we used the most recent 3D Monte Carlo calculations by Honda \textit{et al.}~\cite{Honda:2011nf}. This model is based on the nuclear interaction model JAM used in Particle and Heavy-Ion Transport code System and modified DPMJET-III package.
%Geomagnetic cutoff at Kamioka site is considered.
\subsection{Neutrino oscillations}
Atmosperic neutrino fluxes are modified by neutrino oscillation phenomenon. For computing the flavor transition and survival probabilities we applied the result of the global neutrino oscillation data analysis~\cite{Tortola:2012te} (the relevant values of the neutrino mixing angles, CP violating phase and squared mass splittings are listed in Table~\ref{tab:NOP}).
\begin{table}[!ht]
\begin{center}
\caption{\label{tab:NOP}\small Neutrino mixing parameters for normal (upper rows) and inverse (lower rows) neutrino mass hierarchy.}
\vspace{2mm}
{\small
\begin{tabular}{|c|c|c|c|c|}
\hline
parameter	&best fit	&$1\sigma$	&$2\sigma$	&$3\sigma$\\
\hline
\begin{tabular}{c}
$\Delta{m}^{2}_{21}$\\
$[10^{-5}\mathrm{eV}^{2}]$
\end{tabular}	&$7.62$	&$7.43-7.81$	&$7.27-8.01$	&$7.12-8.20$\\[3mm]
\begin{tabular}{c}
$|\Delta{m}^{2}_{31}|$\\
$[10^{-3}\mathrm{eV}^{2}]$
\end{tabular}
	&\begin{tabular}{c}
		$2.55$\\
		$2.43$
	\end{tabular}	
	&\begin{tabular}{c}
		$2.46-2.61$\\
		$2.37-2.50$
	\end{tabular}
	&\begin{tabular}{c}
		$2.38-2.68$\\
		$2.29-2.58$
	\end{tabular}	
	&\begin{tabular}{c}
		$2.31-2.74$\\
		$2.21-2.64$
	\end{tabular}\\[3mm]
$\sin^{2}\theta_{12}$	&$0.320$	&$0.303-0.336$	&$0.29-0.35$	&$0.27-0.37$\\[3mm]
$\sin^{2}\theta_{23}$	
	&\begin{tabular}{c}
		$0.613$\\
		$0.600$
	\end{tabular}
	&\begin{tabular}{c}
		$0.400-0.461\cup0.573-0.635$\\
		$0.569-0.626$
	\end{tabular}
	&\begin{tabular}{c}
		$0.38-0.66$\\
		$0.39-0.65$
	\end{tabular}
	&\begin{tabular}{c}
		$0.36-0.68$\\
		$0.37-0.67$
	\end{tabular}\\[5mm]
$\sin^{2}\theta_{13}$	
	&\begin{tabular}{c}
		$0.0246$\\
		$0.0250$
	\end{tabular}
	&\begin{tabular}{c}
		$0.0218-0.0275$\\
		$0.0223-0.0276$
	\end{tabular}
	&\begin{tabular}{c}
		$0.019-0.030$\\
		$0.020-0.030$
	\end{tabular}	&$0.017-0.033$\\[3mm]
$\delta$	
	&\begin{tabular}{c}
		$0.80\pi$\\
		$-0.03\pi$
	\end{tabular}
	&$0-2\pi$	&$0-2\pi$	&$0-2\pi$\\[3mm]
\hline
\end{tabular}}
\end{center}
\vspace{-0.5cm}
\end{table}

\subsection{Earth's profile}
Before getting the detector, neutrinos pass through the body of the Earth. Therefore, we had to take into account the neutrino-matter coherent scattering (MSW effect). To solve the Wolfenstein equations we used the method of Ref.~\cite{Naumov:1991ju,Naumov:1991rh}. The density profile in the Earth in the presented calculations is described according to the Two-Layer Earth Model (2LEM)~\cite{Agarwalla:2012uj}.
